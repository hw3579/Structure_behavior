\section{Conclusion}

In conclusion, this experiment allowed us to study the behavior of a portal frame under gradually increasing loads and compare the experimental results with theoretical predictions. We identified the plastic hinges, assessed the frame's behavior, and noted a discrepancy in the collapse load between theory and practice for the deflection ($23.8\%$), which could be attributed to geometric differences and material imperfections.

From observing \autoref{f4}, it can be seen that the W curve changes very dramatically in the initial stage, which may be due to testing errors and a low sampling frequency. Here are some methods for improvement:

$\bullet$ Increase the amount of data for the experiment. The experiment should increase the sampling density to obtain a more accurate point.

$\bullet$ There is experimental randomness. If conditions permit, it is possible to test more identical modular frameworks, and the resulting data will be more reliable.

$\bullet$ Extend the experimental process. If conditions permit, it is possible to try applying larger loads to explore the material's subsequent plastic deformation behavior.