\section{Theory}
\FloatBarrier % Now figures cannot float above section title

The method of calculating the plastic moment is introduced in the powerpoint of the Week 6 lecture. As shown in the figure below \autoref{f1}.

\begin{figure}
    \centering
    \includegraphics[]{./fig/11.png}
    \caption{plastic modules of rectangular section  }
    \label{f1}
\end{figure}

The plastic moment $M_p$ when all points in the section reached
yield stress $\sigma_y$ is caculated by:

\begin{equation} 
    M_p=\sigma_y\frac{bd}{2}(\frac{d}{4}+\frac{d}{4})
\end{equation}

Calculated from the data from \autoref{t1}, we get $M_p=4.33N \cdot m$.

There are three cases of collapse of plasticity of portal frame.

\begin{figure}[htbp]
    \centering
    \includegraphics[width=15cm]{./fig/12.png}
    %\caption{plastic modules of rectangular section  }
    \label{f2}
\end{figure}




























\iffalse

In this section, finite element analysis (ANSYS) is performed on the components.

\subsection*{Analysis}

ANSYS analysis results are shown in Figure \ref{f2}.

\begin{figure}[htbp]
    \centering
    \includegraphics[width=18cm,height=20cm]{./fig/mix2.jpg}
    \caption{Figure of ANSYS analysis}
    \label{f2}
\end{figure}


The finite element analysis in Figure \ref{f2} gives the deformation-position 
curves for different materials under different forces, 
where the maximum deformation of the component can be easily determined.

The data are shown in Table \ref{t4}.

\begin{minipage}[htbp]{\textwidth}
    \makeatletter\def\@captype{table}
    \centering
    \scalebox{1.1}{
    \begin{tabular}{lll} 
        \hline
        Bending Displacement  & Mild Steel    & Aluminium     \\ \hline
        $\delta_{FE\_{1}}(P=50N)$ & 0.1150    & 0.3237    \\
        $\delta_{FE\_{2}}(P=100N)$ & 0.2300    & 0.6473  \\
        $\delta_{FE\_{3}}(P=150N)$ & 0.3451    & 0.9710  \\ \hline          
    \end{tabular}} 
    
    (Unit: mm)
    \caption{FEA results - maximum deformation}
    \label{t4} 
\end{minipage}



\subsection*{Summary}
By using ANSYS analysis, deformation-displacement diagrams were obtained 
for two materials under three different forces, revealing that the maximum 
deformation exists at the midpoint of the element (the point where the forces are applied).

The comparison also shows that even under theoretical conditions, the deformation of aluminium is larger than that of mild steel under the same forces.
\fi