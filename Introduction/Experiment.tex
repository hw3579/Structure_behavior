\section{Experiment}
\FloatBarrier % Now figures cannot float above section title

The frame is made of mild steel and has a uniform rectangular cross-section. Also, The rig is equipped with two gauges to monitor the horizontal deflection of the beam and its central vertical deflection. The yield strenth of the steel is 250MPa.

The experimental parameters are shown in the following table \autoref{t1}.

\begin{minipage}[htbp]{\textwidth}
    \makeatletter\def\@captype{table}
    \centering
    \scalebox{1}{
        \begin{tabular}{lllll}
            \hline
            Data type &Height & Length & Thickness & Width \\ \hline
            Theory&200    & 300    & 3.00      & 12.00 \\ 
            Actual&200    & 304    & 3.27      & 12.97 \\ \hline
            \end{tabular}} 

    (Unit: mm)
    \caption{Experiment parameters}
    \label{t1} 
\end{minipage}


The experiment entails the following steps:
\begin{enumerate}
    \item Measure and record the dimensions of the frame and its cross-section.

    \item Ensure the loading rig is in proper working condition and inspect cables for damage.
    
    \item Zero the force and displacement readings while the rig is still unloaded.
    
    \item Gradually apply increasing horizontal and vertical loads to the frame in increments of 10 N. The relationships between P and W are $y=x$\label{ee1}.
    
    \item Record the applied forces and corresponding deflections for each increment.
    
    \item Continue the process until a plastic failure occurs, and observe the formation of plastic hinges and position.
    
    \item Unload the frame and identify the locations of plastic hinges by observing permanent rotation in the joints.
    
\end{enumerate}

Here are the recorded data \autoref{t2}.

\begin{minipage}[htbp]{\textwidth}
    \makeatletter\def\@captype{table}
    \centering
    \scalebox{0.85}{
        \begin{tabular}{ccccccccccccccc}
            \hline
            \multirow{2}{*}{Force(N)}    & Vertical   & 10   & 20   & 30   & 40   & 50   & 60   & 70   & 80   & 90   & 100  & 110  & 120   & 130   \\
                                      & Horizontal & 10   & 20   & 30   & 40   & 50   & 60   & 70   & 80   & 90   & 100  & 110  & 120   & 130   \\ \hline
            \multirow{2}{*}{Distance(mm)} & Vertical   & 0.06 & 0.05 & 0.1  & 1.42 & 2.13 & 2.26 & 2.8  & 3.22 & 3.62 & 4.28 & 5.23 & 7.46  & 12.88 \\
                                      & Horizontal & 0.88 & 1.58 & 2.15 & 2.99 & 4.04 & 4.44 & 5.06 & 6.48 & 7.63 & 8.55 & 9.31 & 15.05 & 23.08 \\ \hline
            \end{tabular}} 
    \caption{Record data}
    \label{t2} 
\end{minipage}
