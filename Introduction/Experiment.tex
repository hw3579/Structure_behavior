\section{Experiment}
\FloatBarrier % Now figures cannot float above section title

The frame is made of mild steel and has a uniform rectangular cross-section. Also, The rig is equipped with two gauges to monitor the horizontal deflection of the beam and its central vertical deflection.

The experimental parameters are shown in the following table.\ref{t1}

\begin{minipage}[htbp]{\textwidth}
    \makeatletter\def\@captype{table}
    \centering
    \scalebox{1}{
        \begin{tabular}{lllll}
            \hline
            Data type &Height & Length & Thickness & Width \\ \hline
            Theory&200    & 300    & 3.3      & 13 \\ 
            Actual&200    & 304    & 3.27      & 12.97 \\ \hline
            \end{tabular}} 

    (Unit: mm)
    \caption{Experiment parameters}
    \label{t1} 
\end{minipage}


The experiment entails the following steps:
\begin{enumerate}
    \item Measure and record the dimensions of the frame and its cross-section.

    \item Ensure the loading rig is in proper working condition and inspect cables for damage.
    
    \item Zero the force and displacement readings while the rig is still unloaded.
    
    \item Gradually apply increasing horizontal and vertical loads to the frame in increments of 10 N.
    
    \item Record the applied forces and corresponding deflections for each increment.
    
    \item Continue the process until a plastic failure occurs, and observe the formation of plastic hinges and position.
    
    \item Unload the frame and identify the locations of plastic hinges by observing permanent rotation in the joints.
    
\end{enumerate}


\begin{minipage}[htbp]{\textwidth}
    \makeatletter\def\@captype{table}
    \centering
    \scalebox{0.85}{
        \begin{tabular}{ccccccccccccccc}
            \hline
            \multirow{2}{*}{Force(N)}    & Vertical   & 10   & 20   & 30   & 40   & 50   & 60   & 70   & 80   & 90   & 100  & 110  & 120   & 130   \\
                                      & Horizontal & 10   & 20   & 30   & 40   & 50   & 60   & 70   & 80   & 90   & 100  & 110  & 120   & 130   \\ \hline
            \multirow{2}{*}{Distance(mm)} & Vertical   & 0.06 & 0.05 & 0.1  & 1.42 & 2.13 & 2.26 & 2.8  & 3.22 & 3.62 & 4.28 & 5.23 & 7.46  & 12.88 \\
                                      & Horizontal & 0.88 & 1.58 & 2.15 & 2.99 & 4.04 & 4.44 & 5.06 & 6.48 & 7.63 & 8.55 & 9.31 & 15.05 & 23.08 \\ \hline
            \end{tabular}} 
    \caption{Record data}
    \label{t1} 
\end{minipage}



\iffalse
In this section, the deformation of each material under different 
forces can be calculated from the data obtained in section A.

\subsection*{Analysis}

We know
\begin{equation} 
    \delta_{max}=\frac{PL^3}{48EI}=\frac{P*0.1^3}{48*E_{exp}*4.5*10^{-11}}
\end{equation}

Using the $E_{exp}$(172.67GPa and 63.75GPa) in different material(Mild Steel and Aluminium)
 with different force(50N, 100N, 150N) in Table \ref{t2}, the data in Table \ref{t3} can be calculated.

\begin{minipage}[htbp]{\textwidth}
    \makeatletter\def\@captype{table}
    \centering
    \scalebox{1.1}{
    \begin{tabular}{lll} 
        \hline
        Bending Displacement  & Mild Steel    & Aluminium     \\ \hline
        $\delta_{AN\_{1}}(P=50N)$ & 0.1341    & 0.3631    \\
        $\delta_{AN\_{2}}(P=100N)$ & 0.2681     & 0.7262   \\
        $\delta_{AN\_{3}}(P=150N)$ & 0.4022     & 1.0893  \\ \hline          
    \end{tabular}} 
    
    (Unit: mm)
    \caption{Experimental results - maximum deformation}
    \label{t3} 
\end{minipage}

\subsection*{Summary}

Bringing the average modulus of elasticity into the equation enables 
a more accurate calculation of the deformation of the material under 
different forces and helps to reduce experimental errors.
\fi